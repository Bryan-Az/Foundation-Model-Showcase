\documentclass{beamer}
\usetheme{Boadilla}

\title{A Guide to Building a Full-Stack Live Audio Chat Room App}
\author{Alexis Ambriz, CMPE 297}
\date{\today}

\begin{document}

\begin{frame}
    \titlepage
\end{frame}

\section{Introduction}

\begin{frame}
    \frametitle{Introduction}
    \begin{itemize}
        \item Overview of the project
        \item Purpose of the live audio chat room app
        \item Technologies used: ReactJS, NodeJS, Stream API
    \end{itemize}
\end{frame}

\begin{frame}
    \frametitle{Project Overview}
    Imagine a platform like Clubhouse or Twitter Spaces; Users can create and join audio chat rooms!
    This involves real-time audio interactions and discussions.
\end{frame}

\begin{frame}
    \frametitle{Technologies Used}
    Important frameworks and APIs used in the project:
    \begin{itemize}
        \item \textbf{ReactJS}: For building the frontend user interface
        \item \textbf{NodeJS}: For the backend server and API
        \item \textbf{Stream API}: For managing audio calls and user interactions
    \end{itemize}
\end{frame}

\section{Setting Up the Project}

\begin{frame}
    \frametitle{Setting Up the Project}
    Steps involved in setting up the project:
    \begin{itemize}
        \item Creating a new app on GetStream.io
        \item Installing necessary packages: React, NodeJS, Express, etc.
        \item Project directory structure
    \end{itemize}
\end{frame}

\begin{frame}
    \frametitle{GetStream.io Setup}
    GetStream.io is a platform that provides APIs for building in-app chat, video, and audio features.
    To set up the project:
    \begin{itemize}
        \item Sign up for a free account on GetStream.io
        \item Create a new app and obtain API keys
        \item GetStream.io provides APIs for building in-app chat, video, and audio features
    \end{itemize}
\end{frame}

\begin{frame}
    \frametitle{Package Installation}
    \begin{itemize}
        \item Use npm or yarn to install required packages
        \item Examples: `react`, `react-dom`, `node-fetch`, `express`
        \item These packages provide essential functionalities for building the app
    \end{itemize}
\end{frame}

\begin{frame}
    \frametitle{Project Structure}
    \begin{itemize}
        \item Organize your project into clear folders
        \item Example: `frontend`, `backend`, `shared`
        \item This structure helps keep the code organized and maintainable
    \end{itemize}
\end{frame}

\section{Backend Development}

\begin{frame}
    \frametitle{Backend Development}
    \begin{itemize}
        \item NodeJS server setup
        \item ExpressJS for handling HTTP requests
        \item Implementing routes for user authentication, room creation, and joining
        \item Integrating Stream API for audio calls and user interactions
    \end{itemize}
\end{frame}

\begin{frame}
    \frametitle{NodeJS Server Setup}
    \begin{itemize}
        \item Create a new NodeJS project
        \item Initialize the project using `npm init` or `yarn init`
        \item Set up the `server.js` file as the entry point
    \end{itemize}
\end{frame}

\begin{frame}
    \frametitle{ExpressJS for HTTP Requests}
    \begin{itemize}
        \item Integrate the ExpressJS framework
        \item Define routes to handle different HTTP requests
        \item Example: `/api/users`, `/api/rooms`, `/api/join-room`
    \end{itemize}
\end{frame}

\begin{frame}
    \frametitle{Integrating Stream API}
    \begin{itemize}
        \item Use the Stream API to manage audio calls, user interactions, and more
        \item Create users, channels (rooms), and manage audio streams
        \item Use the `node-fetch` package to make HTTP requests to the Stream API
    \end{itemize}
\end{frame}

\section{Frontend Development}

\begin{frame}
    \frametitle{Frontend Development}
    \begin{itemize}
        \item Creating the React app
        \item User Context for managing user state
        \item Main component for authentication and navigation
        \item Room component for displaying room information and audio controls
        \item Integrating Stream JavaScript SDK for audio management
    \end{itemize}
\end{frame}

\begin{frame}
    \frametitle{Creating the React App}
    \begin{itemize}
        \item Use a tool like Create React App to generate a new React project
        \item Set up the basic structure and components
    \end{itemize}
\end{frame}

\begin{frame}
    \frametitle{User Context}
    \begin{itemize}
        \item Create a React Context to store and share user information
        \item Pass the context to components that need access to user data
    \end{itemize}
\end{frame}

\begin{frame}
    \frametitle{Main Component}
    \begin{itemize}
        \item The main component handles user authentication and navigation
        \item It renders different components based on the user's state
    \end{itemize}
\end{frame}

\begin{frame}
    \frametitle{Room Component}
    \begin{itemize}
        \item The Room component displays the content of a specific room
        \item It shows room information, participant list, and audio controls
        \item Uses the Stream JavaScript SDK to manage audio streams
    \end{itemize}
\end{frame}

\begin{frame}
    \frametitle{Integrating Stream JavaScript SDK}
    \begin{itemize}
        \item Use the Stream JavaScript SDK to connect to the Stream service
        \item Manage audio streams, participant list, and other functionalities
    \end{itemize}
\end{frame}

\section{Conclusion}

\begin{frame}
    \frametitle{Conclusion}
    \begin{itemize}
        \item Recap of the key steps involved
        \item Potential applications and future enhancements
        \item Encourage viewers to explore and build upon the project
    \end{itemize}
\end{frame}

\end{document}